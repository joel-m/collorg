\documentclass[9pt,a4paper]{letter}
\usepackage[landscape,vcentering,dvips]{geometry}
\geometry{papersize={210mm,297mm},total={260mm,180mm}}
\usepackage{lscape}
\usepackage{multicol}
\usepackage{fontspec}
\pagenumbering{gobble}
\columnsep 1cm
\begin{document}
\begin{multicols}{3}
\begin{center}
{\Large Programmation en Collorg}
\end{center}
\rule{0.5cm}{0.1pt} \textbf{\large La commande \texttt{cog}} \hrulefill

Permet de gérer une application collorg (initialisation, mises à jour, ...).

\rule{0.5cm}{0.1pt} Pour créer une nouvelle application \texttt{collorg}
\begin{description}
\item[\texttt{cog init -d my\_app}] Initialise une application. Crée un répertoire correspondant au dépôt \texttt{collorg} de l'application.
\end{description}
\rule{0.5cm}{0.1pt} Dans le dépôt \texttt{collorg} de l'application
\begin{description}
\item[\texttt{cog make}] Propage les modifications apportées à l'application.
\item[\texttt{cog struct}] Affiche la structure de la base de donnée de l'application ou d'une table (\texttt{-t}).
\item[\texttt{cog graph}] Affiche le graph de la base de donnée ou d'un schéma de la base de donnée (\texttt{-s}).
\end{description}

\rule{0.5cm}{0.1pt} \textbf{\large API \texttt{collorg}} \hrulefill

\rule{0.5cm}{0.1pt} Le contrôleur

Permet d'initialiser la connexion à la base de donnée. Exemple de script listant les \texttt{FQTNs} de la base:

\begin{scriptsize}
\begin{verbatim}
#!/usr/bin/env python
#-*- coding: utf-8 -*-

from collorg.controller.controller import Controller
relation = Controller().model.relation

types = relation('collorg.core.data_type')
for type_ in types:
	print(type_.fqtn_)
\end{verbatim}
\end{scriptsize}

\rule{0.5cm}{0.1pt} Les relations

\begin{description}
\item[\texttt{relation.select()}] Déclenche l'extraction des données de la base. Une relation est itérable. L'invocation de \texttt{select} est implicite dans un contexte de liste (ex~: \texttt{for elt in relation:})
\item[\texttt{relation.insert()}] Insert la donnée dans la base.
\item[\texttt{relation.update(nval)}] Met à jour la donnée dans la base avec \texttt{nval}.
\item[\texttt{relation.delete()}] Détruit l'ensemble référencé de la base.
\item[\texttt{not relation.is_empty()}] retourne \texttt{True} si l'ensemble défini contient au moins un élément.
\end{description}

\rule{0.5cm}{0.1pt} Les opérateurs algébriques sur les relations

Soient \texttt{rel\_a} et \texttt{rel\_b} deux relations portant sur un même ensemble.

\begin{tabular}{|l|l|l|}
\hline
Python & & Set \\
\hline
$rel\_a + rel\_b$ & $\Longleftrightarrow$ & $rel\_a \cup rel\_b$ \\
$rel\_a * rel\_b$ & $\Longleftrightarrow$ & $rel\_a \cap rel\_b$ \\
$rel\_a - rel\_b$ & $\Longleftrightarrow$ & $rel\_a \backslash rel\_b$ \\
\hline
\end{tabular}

\rule{0.5cm}{0.1pt} Les attributs

\rule{0.5cm}{0.1pt} \textbf{\large Templates \texttt{collorg}} \hrulefill

\rule{0.5cm}{0.1pt} Les balises du langage de templates

\begin{description}
\item[\texttt{\# >>>}] Entrée dans du code python.
\item[\texttt{\# ---}] Zone d'affichage et marque d'indentation hors code python. Le code qui suit sera affiché tel quel.
\item[\texttt{\# +++ <variable>}] Zone d'affichage. Le code qui suit (jusqu'à la balise \texttt{\# --- <variable>} est stocké tel quel (pour affichage) dans la variable \texttt{<variable>}
\item[\texttt{\{\% code\_python \%\}}] Code embarqué dans une zone d'affichage.
\end{description}

\rule{0.5cm}{0.1pt} Exemple de template~:

La template qui suit reçoit un argument \texttt{names} contenant une liste de noms. Si la liste n'est pas vide, elle l'affiche au format \texttt{HTML}.
\begin{scriptsize}
\begin{verbatim}
# >>>
names = kwargs['names']
if len(names):
    # ---
    <ul>
    # >>>
    for name in names:
        # ---
        <li></li>
    # ---
    </ul>
\end{verbatim}
\end{scriptsize}

\end{multicols}
\end{document}
